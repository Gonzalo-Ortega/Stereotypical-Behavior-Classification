
(Provisional)

El objetivo de este trabajo es hacer una breve introducción a ciertas técnicas de aprendizaje automático aplicables al procesado de datos en el análisis de experimentos con animales. Para ello el trabajo se divide en dos capítulos principales.

En el primero abordaremos el estudio teórico de ciertos algoritmos de agrupamiento, reducción de dimensionalidad y de clasificación supervisada. Todo ello desde un punto de vista teórico con ejemplos fáciles de visualizar para ayudar al lector a comprender en profundidad los métodos.

En el segundo capítulo, desarrollaremos paso a paso el análisis real ejercido durante la estancia en el Instituto Cajal. El principal objetivo durante estos meses ha sido buscar la mejor forma de, dados una serie de videos de ratones en su caja, detectar automáticamente diferentes tipos de comportamientos. El fin último del proyecto es estudiar la posibilidad de clasificar dos grupos de animales, control y medicados, basándose en los comportamientos estereotipados.

Para el preprocesado de los videos se ha utilizado la herramienta DeepLabCut, que mediante el uso de redes neuronales y una relativamente pequeña muestra de entrenamiento computa una estimación de la posición de los animales en cada uno de los fotogramas. Tras el preprocesado haremos una revisión de los artículos científicos que han motivado las diferentes técnicas de agrupamiento que hemos acabado utilizando, y finalizaremos desarrollando cada una de ellas a la par que desglosando el código en Python utilizado en cada apartado.

