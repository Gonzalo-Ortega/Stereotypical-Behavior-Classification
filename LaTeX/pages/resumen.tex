A lo largo de la historia, en ciencia en general y en Neurociencia de sistemas en particular, se han realizado todo tipo de experimentos en los cuales los experimentadores tenían que llevar un control manual del proceso, realizando mediciones limitadas sujetas a la interpretación y sesgo de cada experimentador, que necesitaban ser analizadas individualmente. Con el reciente auge de los métodos de aprendizaje automático, ahora es posible recopilar y analizar muchos más datos con un mismo presupuesto, utilizando diversas técnicas para analizar y obtener resultados de los experimentos de forma automática.

Este trabajo se enfoca en distinguir de forma automática los diferentes tipos de comportamientos que realiza un animal en su caja hábitat a lo largo de una sesión de video. Además, busca diferenciar animales control de aquellos que están bajo los efectos de un bloqueador de NMDA. Para ello, primero se realiza una introducción teórica a algoritmos de aprendizaje no supervisado: PCA, agrupación por k-medias, agrupación aglomerada, y agrupación por afinidad; y a algoritmos de aprendizaje supervisado: redes neuronales secuenciales y convolucionales. Posteriormente, se aplican estos algoritmos a datos de la postura corporal de animales a lo largo de sesiones de video procesadas con DeepLabCut, una herramienta de código abierto basada en el aprendizaje profundo, diseñada para rastrear el comportamiento animal con alta precisión.


\mbox{} \bigskip

\noindent \textbf{Palabras clave}:
\begin{compactitem}
    \item Neurociencia de sistemas
    \item Aprendizaje automático
    \item PCA
    \item k-medias
    \item Agrupación aglomerativa
    \item Agrupación por afinidad
    \item Red neuronal secuencial
    \item Red neuronal convolucional
    \item DeepCutLab
    \item Python
\end{compactitem}