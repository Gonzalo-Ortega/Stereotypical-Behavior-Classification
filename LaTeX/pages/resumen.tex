Este trabajo trata de distinguir de forma automática, mediante la clasificación de los comportamientos o de forma directa, si un animal está bajo los efectos de un bloqueador de NMDA. Para ello, primero hace una introducción teórica a algoritmos de aprendizaje no supervisado de PCA, de agrupación por k-medias, y de agrupación aglomerada, y a los algoritmos de aprendizaje supervisado de las redes neuronales secuenciales y convolucionales. Posteriormente, se aplican estos algoritmos a datos de la postura corporal de animales a lo largo de sesiones de video procesadas con DeepCutLab.

\mbox{} \bigskip

\noindent \textbf{Palabras clave}:
\begin{compactitem}
    \item Neurociencia de sistemas
    \item Aprendizaje automático
    \item DeepCutLab
\end{compactitem}