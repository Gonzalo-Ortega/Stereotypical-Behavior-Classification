\section{Preprocesado de datos}

\subsection{DeepLabCut}

\subsection{Filtrado e interpolación}

\subsection{Computo de variables a analizar}

\newpage
\section{Revisión de artículos}

\newpage
\section{Procesado de datos}

\subsection{Matriz de similitud}

\subsection{Agrupamiento por afinidad}

\subsection{Reducción de dimensionalidad para visualización}

\subsection{Entrenamiento de una red neuronal (*por hacer*)}


% \section{Pruebas \LaTeX}
% \subsection{Código}
% 
% \begin{mypython}[caption={Titulo del algoritmo/código.},label={alg:etiqueta}]
% def sum_list_limits_1(a, lower, upper):
%     if lower > upper:
%         return 0
%     else:
%         return a[upper] + sum_list_limits_1(a, lower, upper - 1)
% \end{mypython}
% 
% El código~\ref{alg:etiqueta} es un ejemplo en Python.
% 
% 
% 
% \begin{algorithm}[H]
% \begin{algorithmic}[1]
% \STATE $\forall i \in V$, \ let $i$ be an isolated community
% \STATE $o=permutation(V)$
% \FOR{$k \ \in \ o$}
% \STATE search in $A$ all the neighbours of $k$, $j$
% \STATE $\forall j$, calculate $\Delta Q_k(j)$ in matrix $\mathcal{M}$
% \STATE $j^*=\{ \ j \ | \ \Delta Q_k(j^*)=\max_j\{Q_k(j)\} \ \}$
% \IF{$\Delta Q_k(j^*)>0$}
% \STATE{Move node $k$ to $j^*$ 's community}
% \ELSE
% \STATE{$k$ remains in its community}
% \ENDIF
% \ENDFOR
% \end{algorithmic}\caption{\textit{Additional Louvain} \textbf{input}=$\left(A, \ \mathcal{M}\right)$ \textbf{output}=$P$}
% \label{alg:AdditionalLouvain}
% \end{algorithm}
% 
% En el algoritmo~\ref{alg:AdditionalLouvain} aparece un ejemplo en pseudocódigo.