\chapter{Conclusiones}

Los algoritmos de aprendizaje automático son una poderosa herramienta para analizar todo tipo de datos para multitud de disciplinas diferentes. En este trabajo hemos hecho una revisión de algunos de estos algoritmos explicando su funcionamiento y aplicándolos a un caso de uso en neurociencia de sistemas.

Hemos visto los dos usos principales de los algoritmos de aprendizaje no supervisados: la reducción de dimensionalidad de los datos y la agrupación de estos sin contar con una base de datos etiquetados previamente. Posteriormente, hemos introducido también los algoritmos de aprendizaje supervisado, dando la base matemática de las redes neuronales secuenciales y comentando las cualidades de las redes convolucionales.

El objetivo de este trabajo era usar este tipo de algoritmos para, dada la postura corporal de un ratón a lo largo de una sesión de video, diferenciar automáticamente diferentes comportamientos e identificar si un animal está bajo los efectos de un bloqueador de NMDA. Para ello, hemos partido de los datos de videos de sesiones procesados por DeepLabCut, obteniendo la postura corporal de cada ratón en cada uno de los fotogramas de video. A partir de estos datos, hemos utilizado tres técnicas de análisis diferentes: una clasificación manual de comportamientos y agrupaciones tanto supervisadas como no supervisadas.

La clasificación manual ha funcionado para extraer ciertos comportamientos claramente diferenciables con los puntos de la postura corporal que hemos rastreado. Gracias a ella hemos identificado cuando el animal estaba en movimiento y cuando en reposo, el ángulo que formaba el animal en cada instante, y cuando se ponía a dos patas apoyándose sobre la caja. Pese a funcionar en estos casos, la clasificación manual es un método poco escalable
para identificar comportamientos más sofisticados, ya que para cada nuevo comportamiento a identificar hay que pensar en un algoritmo concreto que pueda diferenciar dicho comportamiento de otros. Además, no es posible utilizar estos algoritmos para diferenciar sesiones de animales tratados de animales no tratados, ya que no conocemos las características visuales que diferencian unos de otros.

Este tipo de algoritmos se pueden enmarcar bajo el término de inteligencia artificial rudimentaria, tomando decisiones en base a condiciones e instrucciones previamente codificadas para realizar esa tarea concreta. Para salvar estos problemas, hemos utilizado algoritmos de aprendizaje automático, otro subconjunto de algoritmos de inteligencia artificial, los cuales analizan multitud de datos para tomar decisiones de forma autónoma.

Primero hemos utilizado algoritmos de aprendizaje no supervisado, los cuales  nos permiten prescindir de datos de entrenamiento, facilitando su aplicación. Para diferenciar el tipo de tratamiento hemos utilizado los algoritmos de k-medias y de agrupación aglomerativa, consiguiendo una tasa de aciertos en la clasificación de nuestros datos del 51,85\% y 58,02\% respectivamente. Ninguno de los dos resultados es muy satisfactorio, lo que puede deberse a varios factores. En ambas agrupaciones hemos utilizado la distancia euclídea como medida de similitud entre los puntos multidimensionales, sin embargo, quizás no sea la medida más apropiada a la luz de los resultados. Se podría repetir el análisis utilizando otro tipo de medidas de similitud, o cambiando la cantidad o la localización de los puntos corporales que se quieren rastrear.

El algoritmo de k-medias y de agrupación aglomerativa necesitan de antemano el número de grupos que se desea realizar. Como para analizar los comportamientos no sabemos cuantos diferentes realiza el animal, hemos utilizado el algoritmo de agrupación por afinidad, que determina autónomamente cuantos grupos realizar. El problema que ha surgido de la utilización de este algoritmo es que es complejo verificar su funcionamiento. No ha generado el mismo número de grupos para todos los animales, y tampoco ha formado algún patrón que hayamos sido capaces de reconocer entre los diferentes tipos de sesiones. A fin de cuentas, ese es el problema al que se enfrentan los algoritmos no supervisados: realizan clasificaciones de los datos, pero no siempre el tipo clasificación que estábamos buscando.

Finalmente hemos utilizado técnicas de aprendizaje automático supervisado diseñado una red neuronal convolucional para diferenciar los tipos de tratamiento. No podemos utilizar esta técnica para diferenciar tipos de comportamientos porque no contamos con una base de datos de comportamientos etiquetados que nos sirvan para entrenar a nuestra red. Tras un proceso de validación cruzada, nuestra red ha obtenido una tasa de aciertos media del 58,27\%, lejos de ser un resultado reseñable. Esto puede ser debido a muchos factores. Al igual que para el análisis con algoritmos no supervisados, la elección de los puntos corporales a rastrear puede no ser la más apropiada, a lo que hay que sumar que hemos contado con una base de datos relativamente pequeña en comparación a la cantidad de datos que suelen utilizarse para entrenar redes en otras disciplinas. También puede ser que la configuración de capas y entradas de nuestra red no sea la más apropiada para el problema, que las funciones de activación no sean las más indicadas, o que directamente el tipo de red elegida no sea la más adecuada para el problema.

Trabajos futuros pueden tratar de probar otras configuraciones de red neuronal, variando los parámetros y su composición. Además, pueden entrenar su red con una cantidad muco mayor de datos, sobre los archivos de video directamente, o con otro tipo de redes o combinación de ellas.