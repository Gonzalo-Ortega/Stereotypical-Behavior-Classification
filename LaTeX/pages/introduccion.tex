
(Provisional)

La Neurociencia es la disciplina científica que estudia el sistema nervioso y todas sus componentes. Para el estudio del funcionamiento del cerebro pueden utilizarse multitud de técnicas diferentes, pero para evitar la intrusión en el sujeto de estudio de muchas de ellas, muchos experimentos e investigaciones se basan en el estudio del comportamiento. Es la etología la encargada de este tipo de estudio en animales y en la cual, históricamente, la categorización de comportamientos dependía fuertemente de la observación manual. Eso ha conllevado siempre una excesiva cantidad de tiempo y las desventajas de los posibles errores de percepción humanos, limitando la cantidad y la calidad de los análisis a realizar.

Gracias a los recientes avances en algoritmos de aprendizaje automático, hoy en día es posible computar multitud de datos de forma simultánea, agilizando y automatizando análisis como el ilustrado previamente.

Este Trabajo de Fin de Grado se ha realizado a lo largo de una estancia de prácticas académicas en el Jercog's Team, un equipo de investigación perteneciente al Instituto Cajal, instituto de Neurociencia del Congreso Superior de Investigaciones Científicas. El equipo está centrado en el estudio de la memoria mediante experimentos con ratones, y uno de sus proyectos abiertos consistía en poder elaborar una herramienta para clasificar automáticamente los comportamientos estereotipados de los ratones. Estos son comportamientos cortos, repetitivos y con cierta tendencia a generar patrones, tales como rascarse, caminar en círculos o levantarse a dos patas, ejecutados sin ninguna finalidad, y usualmente inducidos por estar en entornos cerrados y artificiales.

Durante la estancia se han valorado multitud de técnicas de análisis y procesado de datos, haciendo hincapié en métodos de aprendizaje automático para tratar de completar el proyecto gracias a los últimos avances en computación.

(Para realizar este trabajo se han consultado los libros \cite[Introduction to Machine Learning with Python]{machine} \cite[Deep Learning with PyTorch]{pytorch} y se han tomado ideas y procedimientos de los artículos \cite{deeplabcut} y más).