\chapter{Introducción}
\pagenumbering{arabic}
\setcounter{page}{1}

Este documento recoge el Trabajo de Fin de Grado (TFG) de Gonzalo Ortega Carpintero, incluido en el itinerario del grado en Ingeniería del \textit{Software} de la Universidad Rey Juan Carlos.

En este primer capítulo se presenta la motivación que ha dado lugar a la idea y desarrollo de este TFG en la Sección \ref{sec:motivacion}. En la Sección \ref{sec:objetivos} se introduce el objetivo principal del trabajo y se enumeran los objetivos parciales que se han propuesto para conseguirlo. En la Sección \ref{sec:metodología} se hace un repaso a la metodología seguida para realizar el trabajo y finalmente, en la Sección \ref{sec:herramientas} se introducen las herramientas y bibliotecas que se han utilizado.

\section{Motivación} \label{sec:motivacion}

La Neurociencia de sistemas es una rama de la neurociencia que estudia cómo los diferentes componentes del sistema nervioso, desde las neuronas individuales hasta las redes neuronales, interactúan y funcionan conjuntamente para producir el comportamiento y los procesos mentales \cite{theoretical-neuro}. Para el estudio del funcionamiento del cerebro pueden utilizarse multitud de técnicas diferentes, pero para evitar la intrusión en el sujeto de estudio de muchas de ellas, muchos experimentos e investigaciones se basan en el estudio del comportamiento. Es la etología la encargada de este tipo de estudio en animales y en la cual, históricamente, la categorización de comportamientos dependía fuertemente de la observación manual. Eso ha conllevado siempre una excesiva cantidad de tiempo y las desventajas de los posibles errores de percepción humanos, limitando la cantidad y la calidad de los análisis a realizar.

Gracias a los recientes avances en algoritmos de aprendizaje automático, hoy en día es posible computar multitud de datos de forma simultánea, agilizando y automatizando análisis como el ilustrado previamente.

Este Trabajo de Fin de Grado se ha realizado a lo largo de una estancia de prácticas académicas en el Jercog's Team, un equipo de investigación perteneciente al Instituto Cajal, instituto de Neurociencia del Consejo Superior de Investigaciones Científicas. El equipo está centrado en el estudio de la memoria mediante experimentos con ratones, y uno de sus proyectos abiertos consistía en poder elaborar una herramienta para clasificar automáticamente los comportamientos estereotipados de los ratones. Estos son comportamientos cortos, repetitivos y con cierta tendencia a generar patrones, tales como rascarse, caminar en círculos o levantarse a dos patas, ejecutados sin ninguna finalidad, y usualmente inducidos por estar en entornos cerrados y artificiales.

Para la ejecución de uno de sus últimos experimentos utilizan ratones tratados con anticuerpos bloqueadores de NMDA (\textit{N-Metil-D-aspartato}), restringiendo la capacidad de los ratones para almacenar memoria a largo plazo. En las sesiones de experimentación los ratones tratados ejercen las mismas tareas que ratones control, tratados con un suero placebo, y en las grabaciones hay ratones de ambos tipos. A ojo de un experimentador, es complejo determinar a ciegas si un ratón está bajo los efectos de NMDA o es control, por lo que es interesante la posibililidad de contar con una herramienta que sea capaz de distinguir unos de otros a partir de los videos de las sesiones.

Durante la estancia se han valorado multitud de técnicas de análisis y procesado de datos, haciendo hincapié en métodos de aprendizaje automático para tratar de completar el proyecto gracias a los últimos avances en computación.

\section{Objetivos} \label{sec:objetivos}
El objetivo de este trabajo es distinguir de forma automática, mediante la clasificación de los comportamientos o de forma directa, si un animal está bajo los efectos de un bloqueador de NMDA.

\subsection{Hipótesis}
Para contrastar nuestro objetivo hemos planteado la siguiente hipótesis:

\textit{Dada la postura corporal de un ratón en su caja a lo largo de una sesión de video, mediante técnicas de aprendizaje automático, es posible diferenciar diferentes comportamientos del ratón, e identificar si está bajo los efectos de un bloqueador de NMDA.}

\subsection{Objetivos parciales}
Con el fin de desglosar el objetivo principal, y de poder corroborar o refutar nuestra hipótesis, hemos planteado varios objetivos parciales.

\begin{itemize}
    \item Realizar una introducción a técnicas de aprendizaje automático aplicables al procesado de datos en el análisis de experimentos con animales.
    \item Preparar y preprocesar datos provenientes de archivos de video para su posterior análisis.
    \item Aplicar algoritmos de agrupación para separar fragmentos de video y clasificarlos en función del comportamiento que esté realizando el animal.
    \item Aplicar algoritmos no supervisados para distinguir animales control de animales tratados.
    \item Aplicar algoritmos supervisados para distinguir animales control de animales tratados.
\end{itemize}


\section{Metodología} \label{sec:metodología}

El trabajo se divide en dos capítulos principales, en el Capítulo \ref{chap:fundamentos-teoricos} abordaremos el estudio teórico de ciertos algoritmos de agrupamiento, reducción de dimensionalidad y de clasificación supervisada. Todo ello desde un punto de vista teórico con ejemplos fáciles de visualizar para ayudar al lector a comprender en profundidad los métodos.

En el Capítulo \ref{chap:analisis-y-experimentacion}, desarrollaremos paso a paso el análisis real ejercido durante la estancia en el Instituto Cajal. El principal objetivo durante estos meses ha sido buscar la mejor forma de, dados una serie de videos de ratones en su caja, detectar automáticamente diferentes tipos de comportamientos. El fin último del proyecto es estudiar la posibilidad de clasificar dos grupos de animales, control y medicados, basándose en los comportamientos estereotipados.
