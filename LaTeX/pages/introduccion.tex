
La Neurociencia es la disciplina científica que estudia el sistema nervioso y todas sus componentes. Para el estudio del funcionamiento del cerebro pueden utilizarse multitud de técnicas diferentes, pero para evitar la intrusión en el sujeto de estudio de muchas de ellas, muchos experimentos e investigaciones se basan en el estudio del comportamiento. Es la etología la encargada de este tipo de estudio en animales y en la cual, históricamente, la categorización de comportamientos dependía fuertemente de la observación manual. Eso ha conllevado siempre una excesiva cantidad de tiempo y las desventajas de los posibles errores de percepción humanos, limitando la cantidad y la calidad de los análisis a realizar.

Gracias a los recientes avances en algoritmos de aprendizaje automático, hoy en día es posible computar multitud de datos de forma simultánea, agilizando y automatizando análisis como el ilustrado previamente.

Este Trabajo de Fin de Grado se ha realizado a lo largo de una estancia de prácticas académicas en el Jercog's Team, un equipo de investigación perteneciente al Instituto Cajal, instituto de Neurociencia del Congreso Superior de Investigaciones Científicas. El equipo está centrado en el estudio de la memoria mediante experimentos con ratones, y uno de sus proyectos abiertos consistía en poder elaborar una herramienta para clasificar automáticamente los comportamientos estereotipados de los ratones. Estos son comportamientos cortos, repetitivos y con cierta tendencia a generar patrones, tales como rascarse, caminar en círculos o levantarse a dos patas, ejecutados sin ninguna finalidad, y usualmente inducidos por estar en entornos cerrados y artificiales.

Durante la estancia se han valorado multitud de técnicas de análisis y procesado de datos, haciendo hincapié en métodos de aprendizaje automático para tratar de completar el proyecto gracias a los últimos avances en computación.

\section{Bibliotecas y herramientas utilizadas}

\subsection*{DeepLabCut}
DeepLabCut \cite{deeplabcut} es una herramienta para la estimación 2D y 3D sin marcadores mediante el uso de redes neuronales. Es capaz de identificar de rastrear diferentes partes del cuerpo de múltiples especies realizando todo tipo de comportamientos. Esta ha sido la base de todo nuestro trabajo, ya que todos los videos que hemos analizado han sido procesados en primer lugar por DeepLabCut para rastrear las posiciones de múltiples puntos de los animales a lo largo de los videos de las sesiones. Debido a su importancia, damos una explicación más en detalle de su funcionamiento y de como lo hemos utilizado en la sección \ref{sec:DeepLabCut}.

\subsection*{Google Colab}
Google Colab es una herramienta para realizar cuadernos de Jupyter en línea, y poder ejecutarlos en el \textit{backend} de Google. Estos son documentos que intercalan fragmentos de texto con fragmentos de código ejecutable en Python, así como la salida de las distintas ejecuciones. Todo el codigo realizado para este trabajo ha sido realizado en cuadernos de Colab, y puede ser consultado en los siguientes enlaces:

\begin{table}[h]
    \centering
    \begin{tabular}{|c|p{10cm}|}
        \hline
        \textbf{Ejemplos teóricos} & \href{https://colab.research.google.com/drive/
        1qLLQTCFLZcqExN7qyYyRAT1nm7P8NhuK?usp=sharing}{https://colab.research.google.com/drive/ 1qLLQTCFLZcqExN7qyYyRAT1nm7P8NhuK?usp=sharing} \\
        \hline
        \textbf{DeepCutLab} & \href{https://colab.research.google.com/drive/1dFbmUcfW9el50v0RBCkLIItTD6SF7A3x?usp=sharing}{https://colab.research.google.com/drive/
        1dFbmUcfW9el50v0RBCkLIItTD6SF7A3x?usp=sharing} \\
        \hline
        \textbf{Análisis principal} & \href{https://colab.research.google.com/drive/
        1ak2VpDi-zTnV-uEDp-viEkpa8GFDGuBR?usp=sharing}{https://colab.research.google.com/drive/ 1ak2VpDi-zTnV-uEDp-viEkpa8GFDGuBR?usp=sharing} \\
        \hline
    \end{tabular}
    \caption{Enlaces a cuadernos de Google Colab.}
    \label{tab:colab-links}
\end{table}

\subsection*{Scikit-learn}
Scikit-learn \cite{scikit-learn} es una biblioteca de código abierto de Python que contiene numerosas implementaciones de algoritmos de aprendizaje automático. Hemos usado dichas implementaciones tanto como para la explicación teórica de los algoritmos, como para el análisis real de los datos de los animales.

\subsection*{pandas}
\texttt{pandas} es una biblioteca de manejo de datos mediante tablas denominadas \texttt{DataFrame}. Todos los datos que hemos cargado para ser analizados los hemos guardado como {DataFrames} para poder tener un fácil acceso a todos ellos y a sus variables pudiendo, además, visualizarlos de una forma sencilla.

\subsection*{NumPy}
NumPy es una de las bibliotecas principales de cálculo científico en Python. Proporciona un objeto de \texttt{array} multidimensional y numerosas funciones estadísticas y algebraicas de mucha utilidad.

\subsection*{matplotlib}
\texttt{matplotlib} es una biblioteca para crear figuras en Python. Todas las figuras de representación de datos que aparecen en este trabajo han sido creadas con \texttt{matplotlib}. El código concreto utilizado para generarlas puede consultarse en los cuadernos de Colab.
