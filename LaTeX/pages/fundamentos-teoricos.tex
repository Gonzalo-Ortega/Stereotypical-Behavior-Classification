\section{Métodos no supervisados}

Los algoritmos de aprendizaje automático no supervisados se utilizan cuando no se conoce la salida esperada. Al algoritmo de aprendizaje se le otorgan los datos de entrada y se le pide extraer información de estos datos. Las principales aplicaciones de estos algoritmos, las cuales vamos a aprovechar, son la agrupación de datos y la reducción de dimensionalidad de las variables de los mismos. Esa última es usada principalmente para poder hacer representaciones de datos multidimensionales, los cuales serian complejos de visualizar de otra forma.

La principal pega que pueden tener estos algoritmos es que, si bien no siempre son capaces de identificar conocimiento dados los datos utilizados, cuando lo obtienen, no siempre es el conocimiento que esperábamos obtener. Póngase el ejemplo de un algoritmo que tratase de agrupar rostros de personas iguales. Al no darle a priori ningún tipo de salida de ejemplo, el algoritmo puede acabar clasificando si los rostros están de frente o de lado, no precisamente lo que esperábamos. Es por ello que estos algoritmos cuentan con diversidad de parámetros para ajustarlos a nuestras necesidades, tratando de realizar la agrupación deseada.

En esta sección vamos a estudiar a fondo tres tipos de algoritmos de agrupación: el agrupamiento por k-medias, la agrupación aglomerada, y la agrupación por afinidad. Además, estudiaremos también el principal algoritmo de reducción de dimensiones, el análisis de componentes principales, PCA, de sus siglas en inglés. Los principales ejemplos y explicaciones de los algoritmos han sido inspirados por los dados en el libro \cite[Introduction to Machine Learning with Python]{machine}.

\subsection{Agrupamiento por K-medias}

\subsection{Agrupación aglomerada}

\subsection{Agrupación por afinidad}

\subsection{Análisis de componentes principales}

\section{Métodos supervisados}
\cite[Deep Learning with PyTorch]{pytorch}
\subsection{Red neuronal}